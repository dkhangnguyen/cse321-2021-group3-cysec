\documentclass{article}
\usepackage[utf8]{inputenc}

\title{CySec Report}
\author{
Group 3: Team SERCURITY\\
\texttt{Duc-Khang Nguyen \hspace{0.3in} Jiwan Chong \hspace{0.3in} Jeffrey Begaye }
}

\begin{document}

\maketitle

\textbf{Theme:} A website for Cyber Security news.

\section{Background}

The topic of Cyber Security is an important topic that is relevant currently, and news related to it should be more well-known and accessible to the public eye.

At the moment, there are plenty of news websites out there, and one of our goals is help avoiding the need to open several sites at the same time, by letting the users look up at all the news from our website. 

We plan to make a website that show articles from several big news sources (e.g., CNN, Fox, MSN, etc.) and categorize that data into an easy-to-read format on our website. 

Our main focus for the website is the user's personalized experience. We plan to let user choose their own custom feed - such as which topics of cyber security and from which sources the news from, etc... as well as some quality-of-life features like saving articles, suggesting related articles,...


\section{Website Analysis}

\subsection{a. List five existing websites that are relevant to the theme and describes what functions the websites have}
\begin{itemize}
    \item %%%%%%%%%%%%%%%%%%%% Type information about website 1 under this line %%%%%%%%%%%%%%%%%%%
    \textbf{http://www.cnn.com/:}
    Major new site that is well-known and established.  The front page of the website displays articles under different categories that they find the most relevant.  The headlines on the front page can also contain images, gifs, and short descriptions.  There is a menubar that displays several general categories.  The possibility to create your own user account.  A search function.  The design of the website is grid based.  Unable to vote/rank/comment on the stories.

    \item %%%%%%%%%%%%%%%%%%%% Type information about website 2 under this line %%%%%%%%%%%%%%%%%%%
    \textbf{https://thehackernews.com/:}
    News site specifically for cyber security news.  Articles are listed on the front page based on most recently published/posted article.  There is also a second column with the "Most Popular" articles although we are not sure how it determines what is popular over other articles.  These articles also contain an image, the date, the author posting the article, and a brief description of the article.  The website has a menubar with categories specific to cyber security as well (data breaches, cyber attacks, vulnerabilities, etc.). Design and layout of the website is column based, which seems less cluttered in comparison to CNN.  There is no feature to create user accounts.  But there is an ability to comment on the articles using Facebook.
    
    \item %%%%%%%%%%%%%%%%%%%% Type information about website 3 under this line %%%%%%%%%%%%%%%%%%%
    \textbf{https://www.reddit.com/r/cybersecurity/:}
    A forum themed cyber security news website.  A very user based interactive website for ranking the popularity of such articles and commenting/discussing the articles.  Very intuitive on how to rank articles based on user upvotes/downvotes.  Forum is usually moderated by selected moderators to curate behaviour and/or spam.  Different ways to display articles based on what the users preferences: hot, new, top (hourly, daily, weekly, monthly, etc.), controversial, rising, etc. Usernames, dates, titles, categories, website names, and user options are all displayed within each post.  The Reddit website itself might be confusing at first for first time users, but frequent users can adjust to it fairly quickly.
    
    \item %%%%%%%%%%%%%%%%%%%% Type information about website 4 under this line %%%%%%%%%%%%%%%%%%%
    \textbf{http://www.securityfocus.com/:} 
    Some of the functions of this website are to feature computer security vulnerabilities. It goes by the name BugTraq and is a full disclosure mailing list
    for detailed discussions and annoucements related to computer security.
    \item %%%%%%%%%%%%%%%%%%%% Type information about website 5 under this line %%%%%%%%%%%%%%%%%%%
    \textbf{http://www.linuxsecurity.com/:}
    Some of the functions of this site feature categories such as HowTos,Distribution Security Advisiories,and News.
         Another function this sign has is a sign in option located on the top right of the page along with a Contrubution button to allow users to donate money.
         The site seems to be well organized with a lot of different information related to cybersecurity. 
         It features links to social media pages and a counter of advisories each week.
\end{itemize}



\subsection{b. List and describe the functions the group would like to implement in the website}

\begin{itemize}
    \item \textbf{Headline and Categorization:} Every article of our website will have headline, which include the title, source, topic tag, and if possible, a quick summary. Our articles will be filtered and categorized using topic tag, which is a keyword that groups articles of the same topic together so users can search for related articles easier.
    
    \item \textbf{Top News:} Our website will have a section for featured news from other sites. This can help users quickly get in the loop of what is popular topic of the week/month.
    
    \item \textbf{User Account:} One of our main-focused feature, we want users with account to be able to customize their new feeds on our website, such as choosing what kind of topics they see everyday or filter out what news sources they want to see articles from, as well as saving their favorite article.
    
    \item \textbf{Easy to Navigate Design:} We want to make our website simple to get around and accessible to anyone, such as the Navigation bar should follow the user instead of staying on top of the page.
    
    \item \textbf{Comment Thread and Rating:} User with (and probably without) account will be able to comment on a news article, as well as rating them, which also helps us determine the quality of the news sources we using, as well as determining the popular articles of the week/month.

    
\end{itemize}


\subsection{c. Use a table to compare your proposed website and the existing websites}



\begin{table}[h!]
    \centering
    \begin{tabular}{|c|c|c|c|c|c|c|}
    \hline
                   &  Our Proposed Website & Web 1 & Web 2 & Web 3 & Web 4 & Web 5 \\
    \hline
    Headline/Top News    &            &       &       &       &       &              \\
    \hline
    Categorization    &            &       &       &       &       &              \\
    \hline
    User Account    &            &       &       &       &       &             \\
    \hline
    Easy to navigate design    &            &       &       &       &       &              \\
    \hline
    Comment Thread and Rating    &            &       &       &       &       &              \\ 
    \hline
    \end{tabular}
\end{table}
V:Able to perform the task\par
X:Unable to perform the task\par
O:Able to perform the task with poor interactive design\par

\end{document}
